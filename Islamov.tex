\documentclass[12pt, twoside]{article}
\usepackage{jmlda}
\newcommand{\hdir}{.}
\usepackage[utf8]{inputenc}
\usepackage[english,russian]{babel}
\usepackage{graphicx}


\begin{document}

\title
    [Анализ свойств ансамбля локально аппроксимирующих моделей] % краткое название; не нужно, если полное название влезает в~колонтитул
    {Анализ свойств ансамбля локально аппроксимирующих моделей}
\author
    [Р.\,И.~Исламов] % список авторов (не более трех) для колонтитула; не нужен, если основной список влезает в колонтитул
    {Р.\,И.~Исламов, А.\,В.~Грабовой, В.\,В.~Стрижов} % основной список авторов, выводимый в оглавление
    [Р.\,И.~Исламов$^1$, А.\,В.~Грабовой$^1$, В.\,В.~Стрижов$^{1}$] % список авторов, выводимый в заголовок; не нужен, если он не отличается от основного
\email
    {islamov.ri@phystech.edu; grabovoy.av@phystech.edu;  strijov@ccas.ru}
%\thanks
%    {Работа выполнена при
%     %частичной
%     финансовой поддержке РФФИ, проекты \No\ \No 00-00-00000 и 00-00-00001.}
\organization
    {$^1$Московский физико-технический институт}
\abstract
    {Данная работа посвящена анализу свойств ансамбля локальных моделей. Для задачи регрессии  предлагается использовать многоуровневый подход, согласно которому множество объектов разбивается на несколько подмножеств и каждому подмножеству соответствует одна локальная модель. Рассматривается задача построения универсального аппроксиматора --- мультимодели, которая представлена в виде совокупности локальных моделей. В качестве решающей функции используется выпуклая комбинация локальных моделей.  Коэффициенты выпуклой комбинации --- шлюзовая функция --- функция, значение которой зависит от объекта, для которого производится предсказание. Такой подход позволяет описывать те выборки, которые затруднительно описывать одной моделью. Для анализа свойств проводится вычислительный эксперимент. В качестве данных используются синтетические и реальные выборки. В данной работе реальные данные представлены выборками из boston house prices dataset, servo dataset.  
	
\bigskip
\noindent
\textbf{Ключевые слова}: \emph {локальная модель; линейные модели; ансамбль моделей.}
}

\maketitle
\linenumbers

\section{Введение}
В данной работе исследуется проблема построения мультимодели --- ансамбля локальных моделей. \textit{Локальная модель} --- модель, которая обрабатывает объекты, находящиеся в определенной связной области в пространстве объектов. В качестве агрегирующей функции используется выпуклая комбинация локальных моделей, при этом веса локальных моделей не постоянны, а зависят от положения объекта в пространстве объектов. 

Подход к мультимоделированию предполагает, что вклад каждой локальной модели в ответ зависит от рассматриваемого объекта. Мультимодель использует шлюзовую функцию, которая определяет значимость предсказания каждой локальной модели, входящей в ансамбль.

В данной работе каждая локальная модель является линейной. В качестве функционала качества рассматривается логарифм правдоподобия модели. Предлагается алгоритм нахождения оптимальных параметров ансамбля и локальных моделей. 

Преимуществом данного подхода является его способность описывать те выборки, которые затруднительно описывать одной моделью, и разбивать выборку в соответствии с выбранными моделями.

Алгоритмы тестировались на синтетических и реальных данных. Реальные данные представляли собой boston house prices и servo datasets. Эксперименты показали преимущество использования многоуровневой модели и смеси моделей по сравнению с использованием одной модели.


В прикладных задачах данные порождены в результате использования нескольких источников, либо гипотеза порождения и вовсе не известна. В таких случаях качество предсказания можно повышать увеличивая количество моделей. Если моделей на самом деле меньше, чем предполагается, то веса лишних моделей будут малы и их вклад
будет несущественен. Этим объясняется актуальность использования мультимоделирования.

\section{Работы по теме}

С момента своего появления мультимодельный подход стал предметом многих исследований. Были предложены различные типы архитектур локальных моделей, такие как SVM \cite{Collobert2002}, Гауссовский процесс \cite{Tresp01mixturesof}  и нейронные сети \cite{Shazeer2017}. Другие работы была сосредоточены на различных конфигурациях, таких как иерархическая структура \cite{NIPS1991_514}, бесконечное число экспертов \cite{Rasmussen} и последовательное добавление экспертов \cite{Aljundi2016}. \cite{garmash-monz-2016-ensemble} предлагает модель ансамбля локальных моделей для машинного перевода. Стробирующая сеть обучается на предварительно обученной модели NMT ансамбля. 

Ансамбль локальных моделей имеет множество приложений в прикладных задачах. Работы \cite{Yumlu2003, Cheung1995, Weigend2000} посвящены применению смеси экспертов в задачах прогнозирования временных
рядов. В работе \cite{article} предложен метод распознавания рукописных цифр.   
 
\bibliographystyle{unsrt}
\bibliography{Islamov}

\end{document}
